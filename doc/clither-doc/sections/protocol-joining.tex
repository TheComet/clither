\subsection{Joining}

\begin{center}
\begin{tikzpicture}[>=latex]
\coordinate (A) at (2,5);
\coordinate (B) at (2,0);
\coordinate (C) at (7,5);
\coordinate (D) at (7,0);
\draw[thick] (A)--(B) (C)--(D);
\draw (A) node[above]{\Large Client};
\draw (C) node[above]{\Large Server};

\coordinate (E) at ($(A)!.1!(B)$);
\coordinate (F) at ($(C)!.25!(D)$);
\coordinate (G) at ($(C)!.35!(D)$);
\coordinate (H) at ($(A)!.5!(B)$);
\coordinate (I) at ($(C)!.5!(D)$);
\coordinate (J) at ($(A)!.65!(B)$);
\coordinate (K) at ($(C)!.65!(D)$);
\coordinate (L) at ($(A)!.8!(B)$);
\coordinate (M) at ($(C)!.8!(D)$);
\coordinate (N) at ($(A)!.95!(B)$);
\draw[->] (E) -- (F) node[midway,sloped,above]{\verb$JOIN_REQUEST$};
\draw[dashed,->] (G) -- (H) node[midway,sloped,above]{\verb$JOIN_REQUEST_ACK$};
\draw[dashed,->] (I) -- (J) node[midway,sloped,above]{\verb$JOIN_DENY_BAD_PROTOCOL$};
\draw[dashed,->] (K) -- (L) node[midway,sloped,above]{\verb$JOIN_DENY_BAD_USERNAME$};
\draw[dashed,->] (M) -- (N) node[midway,sloped,above]{\verb$JOIN_DENY_SERVER_FULL$};
\draw (G) node[right]{Accept?};
\draw (I) node[right]{Incompatible protocol?};
\draw (K) node[right]{\makecell[l]{Username  too long? \\ Profanity?}};
\draw (M) node[right]{Player cap?};
\end{tikzpicture}

\end{center}

The  client  initiates  joining  by  sending  a \verb$MSG_JOIN_REQUEST$ to the
server.  The  join  request is resent periodically to account for  UDP  packet
loss using a resend counter.

\begin{itemize}
\item If the server accepts, it  will  spawn a snake in the world and ``hold''
it in place. Held snakes are not simulated until the server receives the first
command   message   from   the   client.   The   server  will   respond   with
\verb$MSG_JOIN_ACCEPT$ for every \verb$MSG_JOIN_REQUEST$ it receives, even  if
the  snake has already been spawned. The client must therefore  make  sure  to
only  handle  the  initial  accept  message,  and ignore any subsequent accept
messages.
\item The server can deny a join request for a number of reasons. The protocol
version could differ. The  username  could be too long, or contain blacklisted
words. The  server could be full. The server does not expect any more messages
after responding with a deny message.
\end{itemize}

\vspace{1.5em}

\begin{figure}[h]
\verb$MSG_JOIN_REQUEST$\\
\vspace{.25em}\\
\begin{bytefield}[bitwidth=4em,bitformatting=\small]{1}
\bitheader{0-4} \\
\bitbox{1}{major}
\bitbox{1}{minor}
\bitbox{2}{frame}
\bitbox{1}{len}
\bitbox{3}{username}
\bitbox{1}{NULL}
\end{bytefield}


\begin{itemize}
    \item\textbf{major:} Protocol version major.
    \item\textbf{minor:} Protocol version minor.
    \item\textbf{frame:} Current frame number of the client.
    \item\textbf{len:} $strlen()$ of the username (excluding null terminator).
    \item\textbf{username:} UTF-8 encoded username.
    \item\textbf{NULL:} Username string is null terminated.
\end{itemize}
\end{figure}

\vspace{1.5em}

\begin{figure}[h]
\verb$MSG_JOIN_ACCEPT$\\
\vspace{.25em}\\
\begin{bytefield}[bitwidth=3em,bitformatting=\small]{1}
\bitheader{0-13} \\
\bitbox{1}{str}
\bitbox{1}{ntr}
\bitbox{2}{cl frame}
\bitbox{2}{sv frame}
\bitbox{2}{snake ID}
\bitbox{3}{spawn X}
\bitbox{3}{spawn Y}
\end{bytefield}


\begin{itemize}
    \item\textbf{str:} Simulation tick-rate used by the server.
    \item\textbf{ntr:} Networking tick-rate used by the server.
    \item\textbf{cl frame:} Client frame number sent back from \verb$MSG_JOIN_REQUEST$.
    \item\textbf{sv frame:} Current frame number of the server.
    \item\textbf{snake ID:} Unique identifier of the snake spawned by the server.
    \item\textbf{spawn X:} X-coordinate of spawn location in Q19.5 in world space.
    \item\textbf{spawn Y:} Y-coordinate of spawn location in Q19.5 in world space.
\end{itemize}
\end{figure}
