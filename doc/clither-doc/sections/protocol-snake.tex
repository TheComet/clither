\subsection{Snakes}

If another player is in  proximity  of  the  client,  the server begins sending
information of that player's snake to the client.

The server initially sends all \verb$MSG_SNAKE_BEZIER$ packets of that player's
snake  to  the  client. The client can send back \verb$MSG_SNAKE_BEZIER_ACK$ to
acknowledge a knot  has  been  received.  The server keeps track of which knots
need to be  sent  to  a  client, and will only send non-acknowledged knots. The
server   will   also   send   \verb$MSG_SNAKE_USERNAME$   until   it   receives
\verb$MSG_SNAKE_USERNAME_ACK$.

Parallel  to  snake  creation,  the server will send \verb$MSG_SNAKE_HEAD$  and
\verb$MSG_SNAKE_PARAM$ updates whenever the snake moves forwards. These packets
are  lossy  and  the  client  only  needs  the  most recent head position/snake
parameters. The snake  parameters  contain status effects as well as the length
of the snake, which allows the  client  to  remove  stale bezier knots from the
tail end of the snake.

\begin{figure}[H]
    \centering
    \begin{subfigure}{.48\linewidth}
        \centering
        \begin{tikzpicture}[>=latex]
\coordinate (A) at (2,5);
\coordinate (B) at (2,0);
\coordinate (C) at (6,5);
\coordinate (D) at (6,0);
\draw[thick] (A)--(B) (C)--(D);
\draw (A) node[above]{\Large Client};
\draw (C) node[above]{\Large Server};

\coordinate (E) at ($(C)!.1!(D)$);
\coordinate (F) at ($(A)!.25!(B)$);
\coordinate (G) at ($(C)!.2!(D)$);
\coordinate (H) at ($(A)!.35!(B)$);
\coordinate (I) at ($(C)!.3!(D)$);
\coordinate (J) at ($(A)!.45!(B)$);
\coordinate (K) at ($(A)!.55!(B)$);
\coordinate (L) at ($(C)!.7!(D)$);
\coordinate (M) at ($(A)!.65!(B)$);
\coordinate (N) at ($(C)!.8!(D)$);
\coordinate (O) at ($(A)!.75!(B)$);
\coordinate (P) at ($(C)!.9!(D)$);
\draw[->] (E) -- (F) node[midway,sloped,above]{\verb$SNAKE_USERNAME$};
\draw[->] (G) -- (H) node[midway,sloped,above]{\verb$SNAKE_BEZIER 1$};
\draw[->] (I) -- (J) node[midway,sloped,above]{\verb$SNAKE_BEZIER 2$};
\draw[->] (K) -- (L) node[midway,sloped,above]{\verb$SNAKE_USERNAME_ACK$};
\draw[->] (M) -- (N) node[midway,sloped,above]{\verb$SNAKE_BEZIER_ACK 1$};
\draw[->] (O) -- (P) node[midway,sloped,above]{\verb$SNAKE_BEZIER_ACK 2$};
\draw (E) node[right]{Proximity?};
\end{tikzpicture}


        \caption{Snake creation}
    \end{subfigure}
    \begin{subfigure}{.48\linewidth}
        \centering
        \begin{tikzpicture}[>=latex]
\coordinate (A) at (2,5);
\coordinate (B) at (2,0);
\coordinate (C) at (6,5);
\coordinate (D) at (6,0);
\draw[thick] (A)--(B) (C)--(D);
\draw (A) node[above]{\Large Client};
\draw (C) node[above]{\Large Server};

\coordinate (E) at ($(C)!.1!(D)$);
\coordinate (F) at ($(A)!.25!(B)$);
\coordinate (G) at ($(C)!.2!(D)$);
\coordinate (H) at ($(A)!.35!(B)$);
\coordinate (I) at ($(C)!.35!(D)$);
\coordinate (J) at ($(A)!.5!(B)$);
\coordinate (K) at ($(C)!.45!(D)$);
\coordinate (L) at ($(A)!.6!(B)$);
\coordinate (M) at ($(C)!.6!(D)$);
\coordinate (N) at ($(A)!.75!(B)$);
\coordinate (O) at ($(C)!.7!(D)$);
\coordinate (P) at ($(A)!.85!(B)$);
\draw[->] (E) -- (F) node[midway,sloped,above]{\verb$SNAKE_HEAD$};
\draw[->] (G) -- (H) node[midway,sloped,above]{\verb$SNAKE_PARAM$};
\draw[->] (I) -- (J) node[midway,sloped,above]{\verb$SNAKE_HEAD$};
\draw[->] (K) -- (L) node[midway,sloped,above]{\verb$SNAKE_PARAM$};
\draw[->] (M) -- (N) node[midway,sloped,above]{\verb$SNAKE_HEAD$};
\draw[->] (O) -- (P) node[midway,sloped,above]{\verb$SNAKE_PARAM$};
\draw (E) node[right]{frame 1};
\draw (I) node[right]{frame 2};
\draw (M) node[right]{frame 3};
\end{tikzpicture}


        \caption{Snake updates}
    \end{subfigure}
\end{figure}

When  the  snake  goes out of range of the client, the server will simply  stop
sending updates to the client. The  client  can use a timeout to determine when
to  destroy  its  local  copy of the snake. If the snake comes  into  proximity
again, the server will always assume the client has  destroyed  the  snake  and
send everything again.

\begin{figure}[h]
\verb$MSG_SNAKE_USERNAME$
\vspace{.25em}\\

\begin{bytefield}[bitwidth=3em,bitformatting=\small]{1}
    \bitheader{0-1} \\
    \bitbox{1}{len}
    \bitbox{3}{username}
    \bitbox{1}{NULL}
\end{bytefield}

\begin{itemize}
    \item\textbf{len:} $strlen()$ of the username (excluding null terminator).
    \item\textbf{username:} UTF-8 encoded username.
    \item\textbf{NULL:} Username string is null terminated.
\end{itemize}


\end{figure}

\vspace{1.5em}

\begin{figure}[h]
\verb$MSG_SNAKE_BEZIER$
\vspace{.25em}\\

\begin{bytefield}[bitwidth=3em,bitformatting=\small]{1}
    \bitheader{0-12} \\
    \bitbox{2}{snake ID}
    \bitbox{2}{handle ID}
    \bitbox{3}{pos x}
    \bitbox{3}{pos y}
    \bitbox{1}{angle}
    \bitbox{1}{lenf}
    \bitbox{1}{lenb}
\end{bytefield}

\begin{itemize}
    \item\textbf{snake ID:} Unique snake identifier.
    \item\textbf{handle ID:} Handle position in the ring buffer.
    \item\textbf{pos x:} Q19.5 x coordinate of the bezier knot in world space.
    \item\textbf{pos y:} Q19.5 y coordinate of the bezier knot in world space.
    \item\textbf{angle:} Q4.12 angle of the bezier knot.
    \item\textbf{lenf:} Length of the ``forward facing'' bezier handle.
    \item\textbf{lenb:} Length of the ``backwards facing'' bezier handle..
\end{itemize}

\end{figure}


