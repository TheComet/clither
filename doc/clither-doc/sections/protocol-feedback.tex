\subsection{Feedback}

Feedback is the mechanism for controlling  the size of the server-side command
buffer. The  client  will  slightly adjust its simulation speed depending on a
feedback value sent  by  the  server.  This  is  known  as  ``warping  time''.

\begin{center}
\begin{tikzpicture}[>=latex]
\coordinate (A) at (2,5);
\coordinate (B) at (2,0);
\coordinate (C) at (6,5);
\coordinate (D) at (6,0);
\draw[thick] (A)--(B) (C)--(D);
\draw (A) node[above]{\Large Client};
\draw (C) node[above]{\Large Server};

\coordinate (E) at ($(A)!.1!(B)$);
\coordinate (F) at ($(C)!.25!(D)$);
\coordinate (G) at ($(C)!.54!(D)$);
\coordinate (H) at ($(C)!.75!(D)$);
\coordinate (I) at ($(A)!.9!(B)$);
\draw[->] (E) -- (F) node[midway,sloped,above]{\verb$COMMANDS$};
\draw[->] (H) -- (I) node[midway,sloped,above]{\verb$FEEDBACK$};
\draw (G) node[right]{
\begin{bytefield}[bitwidth=4em]{1}
\bitbox{1}{$\text{cmd}_n$} \\
\bitbox{1}{\ldots} \\
\bitbox{1}{$\text{cmd}_2$} \\
\bitbox{1}{$\text{cmd}_1$} \\
\end{bytefield}
};
\end{tikzpicture}


\end{center}

The server would like to keep this buffer  as  small as possible to reduce the
client's latency. However, this increases the risk  of  the  server  having to
predict  the  next frame, because the server  may  not  receive  the  client's
command in time for stepping the simulation. If  this happens, the server will
extrapolate the  client's  next  command.  If the extrapolated command differs
from  the client's actual command, then a misprediction has occurred  and  the
client  will have to rollback and reconcile  once  it  receives  the  server's
authoritative state.

If  the  buffer  size  is  larger,  then  it  is  more  resillient  to network
fluctuations and packet loss.

\vspace{1.5em}

\begin{figure}[h]
\verb$MSG_FEEDBACK$
\vspace{.25em}\\

\begin{bytefield}[bitwidth=3em,bitformatting=\small]{1}
    \bitheader{0-2} \\
    \bitbox{2}{frame}
    \bitbox{1}{diff}
\end{bytefield}

\begin{itemize}
    \item\textbf{diff:}  Difference in simulation-ticks the server would  like  the
    \item\textbf{frame:} Server frame-number when the  message  was sent. client to
    ``warp''  to. If negative, the client will slow down its simulation  speed.  If
    positive, the client will speed up its simulation speed.
\end{itemize}


\end{figure}

